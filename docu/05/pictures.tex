\section{Bilder verwalten}
\subsection{Bild hochladen}
\label{subsec:uploadPic}
Zuerst wird das Album, zu dem Fotos hinzugefügt werden sollen, wie im Kapitel \enquote{\nameref{subsec:modAlbum}} beschrieben.

In der Ansicht sind unten 3 leere Einträge (Zeilen) zum Hinzufügen von weiteren Bildern vorgesehen.
Zuerst muss ein Bild ausgewählt werden. Dazu muss auf den Button \enquote{Browse...} in de 3. Spalte geklickt werden. Im sich öffnenden Dialog wird das hochzuladene Foto ausgewählt.

In der ersten und zweiten Spalte kann optional noch ein Name und eine Beschreibung für das Bild hinterlegt werden.

Durch einen Klick auf den Button \enquote{Save} werden die neuen Bilder in Tartan übernommen.

\subsection{Bildinformationen bearbeiten}
\label{subsec:modPic}
Man verfährt wie in Kapitel \enquote{\nameref{subsec:uploadPic}} beschrieben.

Statt aber in den den unteren 3 Zeilen ein neues Bild hinzuzufügen werden die Informationen in den Zeilen darüber, also der bestehenden Bilder verändert.

Die Änderungen werden durch Klick auf den Button \enquote{Save} übernommen

\subsection{Bild entfernen}
Man verfährt wie in Kapitel \enquote{\nameref{subsec:modPic}}.

Beim zu löschenden Bild wird in der letzen Spalte die Checkbox gesetzt.

Durch drücken des \enquote{Save}-Buttons wird das Bild aus der Datenbank gelöscht.