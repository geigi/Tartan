\section {Alben verwalten}
\subsection{Album anlegen}
Jedes Bild muss genau einem Album zugeordnet sein. Daher muss vor dem Upload von Bildern ein Album angelegt werden.

Dazu drückt man in der Admin-Hauptansicht in der Kategorie \enquote{Photogallery} neben dem Menüpunkt \enquote{Albums} den Button \enquote{Add}

Im sich öffnenden Formular muss zuerst ein Name vergeben werden. Obwohl nicht empfohlen dürfen mehrere Alben den selben Namen besitzen.

Anschließend kann das Album schon mit einem Klick auf \enquote{Save} unten rechts gespeichert werden.
\subsection{Album barbeiten}
\label{subsec:modAlbum}
In der Admin-Hauptseite wird der Eintrag \enquote{Albums} in der Kategorie \enquote{Photogallery} ausgewählt.

Es öffnet sich eine Liste aller Alben. Hier wird das zu bearbeitende Album ausgewählt.
Im folgenden Formular kann das Album bearbeitet werden.

Die Änderungen werden mit einem Klick auf den Button \enquote{Save} unten rechts gespeichert

\subsection{Album löschen}
Man verfährt wie in \enquote{\nameref{subsec:modAlbum}} beschrieben um zur Liste aller Alben zu gelangen. 

Dort wählt man die zu löschenden Alben durch Anwählen der entsprechenden Checkbox aus.

Im DropDown-Menü über der Albenliste wählt man anschießend \enquote{Delete selected albums} aus und bestätigt mit einem Klick auf \enquote{Go}.