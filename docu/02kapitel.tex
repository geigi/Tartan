\chapter{Verwendete Technologien}
\section{UserInterface}
\subsection{jQuery}
\begin{figure}[h]
  \centering
  \includegraphics[width=0.5\textwidth]{images/jQuery_logo.pdf}
\end{figure}
jQuery ist eine JavaScript-Bibliothek, welche Funktionen zur einfachen \textit{Document Object Model} Modifikation beinhaltet. Mit ihr können z.B. Animationen sehr einfach auf Objekte angewandt werden, welche in diesem Projekt auch verwendet werden.

Homepage: \url{http://jquery.com}

\subsection{JScrollPane}
\begin{figure}[h]
  \centering
  \includegraphics[width=0.5\textwidth]{images/jScrollPane_logo.png}
\end{figure}
Die jScrollPane ist ein jQuery Plugin, welches die Scrollbalken der Browser in eine HTML-Struktur umwandelt, welche einfach verändert werden kann. So kann beispielsweise das Aussehen, aber auch das Verhalten der Balken geändert werden. Verwendet wurde dieses Plugin bei der Schnellauswahl von Bildern eines Albums.

Homepage: \url{http://jscrollpane.kelvinluck.com}

\subsection{Font Awesome}
\begin{figure}[h]
  \centering
  \includegraphics[width=0.5\textwidth]{images/fontawesome_logo.png}
\end{figure}
Font Awesome ist eine Sammlung aus skalierbaren Verktorgrafiken. Sie enthält oft benötigte Grafiken wie Icons, Pfeile, Logos (z.B. Facebook). Ein weiterer Vorteil ist, dass die Icons sehr einfach mit CSS verändert werden können. Angepasst werden kann z.B. die Größe, Farbe und der Schatten. Fast jedes Icon dieses Projekts stammen aus dieser Sammlung, die anderen sind selbst erstellt. Da Font Awesome unter der MIT Lizenz steht, können die Icons für fast jedes Projekt verwendet werden.

Homepage: \url{http://fortawesome.github.io/Font-Awesome/}

\subsection{HTML und CSS}
\begin{figure}[h]
  \centering
  \includegraphics[width=0.5\textwidth]{images/html5_css_logo.pdf}
\end{figure}
In diesem Projekt werden aktuellste Funktionen von HTML und CSS verwendet. Dazu gehören z.B. Farbverläufe mit CSS (beim Hintergrund) und Transparenzen. 

Vorteile:
\begin{itemize}
	\item Modernes Aussehen kann sehr einfach erzeugt werden
	\item Übersichtlicher, kurzer und wartungsarmer Quellcode im Gegensatz zu ``Bastellösungen''
	\item Zukunftssicher
\end{itemize}
Nachteile:
\begin{itemize}
	\item Ältere Browser werden nicht unterstützt (siehe Kapitel \ref{besAnforderungen}). Warnseite für Besucher mit zu altem Browser wurde erstellt
\end{itemize}

Der Quellcode wurde mit dem \textit{W3C}-Validator validiert. Der Test ergab keine Fehler.

Homepage: \url{http://validator.w3.org}

\section{Backend}