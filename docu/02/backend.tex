\section{Backend}
\subsection{Django}
\begin{figure}[h]
  \centering
  \includegraphics[width=0.7\textwidth]{images/django-logo-negative.pdf}
  \caption{Django Logo}
\end{figure}
	Django ist ein Web Framework für Python.
\subsection{Python 3}
\begin{figure}[h]
  \centering
  \includegraphics[width=0.7\textwidth]{images/python-logo.pdf}
  \caption{Python Logo}
\end{figure}
	Python ist eine interpretierte, [...] objektorientierte Programmiersprache.\footnote{\citet{pythonFaq}}
\subsection{Pillow}
	Pillow ist ein Fork der Python Image Library (PIL).

	Pillow ermöglicht die einfache Verarbeitung von Bildern in Python. Es kann alle gebräuchlichen und viele exotischere Bildformate lesen und schreiben. Es bietet viele Möglichkeiten zur Bearbeitung von Bildern, beispielsweise Größenänderung, Helligkeits- und Kontrastanpassung und Formatkonvertierungen.

	Da die aktuellste verfügbare Version von PIL bei der Erstellung dieses Projektes bereits über 5 Jahre alt ist wird hier dessen zurzeit aktiv weiterentwickelter Fork Pillow verwendet.